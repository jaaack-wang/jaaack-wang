%-------------------------
% Resume in Latex
% Author : Zhengxiang Wang
% License : MIT
%------------------------

\documentclass[letterpaper,11pt]{article}

\usepackage{latexsym}
\usepackage[empty]{fullpage}
\usepackage{titlesec}
\usepackage{marvosym}
\usepackage[usenames,dvipsnames]{color}
\usepackage{verbatim}
\usepackage{enumitem}
\usepackage[hidelinks]{hyperref}
\usepackage{fancyhdr}
\usepackage[english]{babel}
\usepackage{tabularx}
\usepackage{hyphenat}
\usepackage{fontawesome}
\usepackage{xcolor}
\input{glyphtounicode}


%---------- FONT OPTIONS ----------
% sans-serif
% \usepackage[sfdefault]{FiraSans}
% \usepackage[sfdefault]{roboto}
% \usepackage[sfdefault]{noto-sans}
% \usepackage[default]{sourcesanspro}

% serif
% \usepackage{CormorantGaramond}
% \usepackage{charter}


\pagestyle{fancy}
\fancyhf{} % clear all header and footer fields
\fancyfoot{}
\renewcommand{\headrulewidth}{0pt}
\renewcommand{\footrulewidth}{0pt}

% Adjust margins
\addtolength{\oddsidemargin}{-0.5in}
\addtolength{\evensidemargin}{-0.5in}
\addtolength{\textwidth}{1in}
\addtolength{\topmargin}{-.5in}
\addtolength{\textheight}{1.0in}

\urlstyle{same}

\raggedbottom
\raggedright
\setlength{\tabcolsep}{0in}

% Sections formatting
\titleformat{\section}{
  \vspace{-4pt}\scshape\raggedright\large
}{}{0em}{}[\color{black}\titlerule \vspace{-5pt}]

% Ensure that generate pdf is machine readable/ATS parsable
\pdfgentounicode=1

%-------------------------
% Custom commands

\newcommand{\resumeItem}[1]{
  \item\small{
    {#1 \vspace{-2pt}}
  }
}


\newcommand{\resumeSubheading}[4]{
  \vspace{-2pt}\item
    \begin{tabular*}{0.97\textwidth}[t]{l@{\extracolsep{\fill}}r}
      \textbf{#1} & #2 \\
      \textit{\small#3} & \textit{\small #4} \\
    \end{tabular*}\vspace{-7pt}
}


\newcommand{\resumeSubSubheading}[2]{
    \vspace{-2pt}\item
    \begin{tabular*}{0.97\textwidth}{l@{\extracolsep{\fill}}r}
      \textit{\small#1} & \textit{\small #2} \\
    \end{tabular*}\vspace{-7pt}
}


\newcommand{\resumeEducationHeading}[6]{
  \vspace{-2pt}\item
    \begin{tabular*}{0.97\textwidth}[t]{l@{\extracolsep{\fill}}r}
      \textbf{#1} & #2 \\
      \textit{\small#3} & \textit{\small #4} \\
      \textit{\small#5} & \textit{\small #6} \\
    \end{tabular*}\vspace{-7pt}
}


\newcommand{\resumeProjectHeading}[2]{
    \vspace{-2pt}\item
    \begin{tabular*}{0.97\textwidth}{l@{\extracolsep{\fill}}r}
      \small#1 & #2 \\
    \end{tabular*}\vspace{-7pt}
}

\newcommand{\resumeSubItem}[1]{\resumeItem{#1}\vspace{-4pt}}

\renewcommand\labelitemii{$\vcenter{\hbox{\tiny$\bullet$}}$}

\newcommand{\resumeSubHeadingListStart}{\begin{itemize}[leftmargin=0.15in, label={}]}
\newcommand{\resumeSubHeadingListEnd}{\end{itemize}}
\newcommand{\resumeItemListStart}{\begin{itemize}}
\newcommand{\resumeItemListEnd}{\end{itemize}\vspace{-5pt}}

%-------------------------------------------
%%%%%%  RESUME STARTS HERE  %%%%%%%%%%%%%%%%%%%%%%%%%%%%


\begin{document}

%---------- HEADING ----------

\begin{center}
    \textbf{\Huge \scshape Zhengxiang Wang} \\ \vspace{3pt}
    \small
    \faEnvelope \hspace{.5pt} \href{mailto:jackwang196531@gmail.com}{jackwang196531@gmail.com}
    $|$
    \faHome \hspace{.5pt} \href{https://jaaack-wang.eu.org}{https://jaaack-wang.eu.org}
    $|$
    \faGithub \hspace{.5pt} \href{https://github.com/jaaack-wang}{GitHub}
    $|$
    \faLinkedinSquare \hspace{.5pt} \href{https://www.linkedin.com/in/zhengxiang-wang-560735191/}{LinkedIn}
    
\end{center}


%----------- EDUCATION -----------

\section{Education}
  \resumeSubHeadingListStart
    
    \resumeSubheading
      {Stony Brook University}{Stony Brook, NY}
      {Ph.D. in Computational Linguistics}{Aug 2022 \textbf{--} Present}

    \vspace{3pt}
    \resumeSubheading
      {University of Saskatchewan}{Saskatoon, Canada}
      {M.A. in Applied Linguistics}{Sep 2019 \textbf{--} May 2021}
      
    \vspace{3pt}
    \resumeSubheading
      {Hunan University}{Changsha, China}
      {B.A. in Chinese Language and Literature}{Sep 2015 \textbf{--} Jun 2019}
    
  \resumeSubHeadingListEnd


% ----------- CERTIFICATES -----------

\section{Certificates}
  \resumeSubHeadingListStart
    \small{\item{
        \textbf{Machine Learning Specialization}, Deeplearning.AI \hfill 2022 \\ \vspace{1pt}
        \textbf{Practical Data Science on the AWS Cloud}, Deeplearning.AI \hfill 2022 \\ \vspace{1pt}
        \textbf{Deep Learning Specialization}, Deeplearning.AI \hfill 2021 \\ \vspace{1pt}
        \textbf{Machine Learning}, Stanford Online \hfill 2021 \\ \vspace{1pt}
        \textbf{Deep Learning Based Natural Language Processing}, Baidu PaddlePaddle \hfill 2021
    }} \\ \vspace{-5pt}
  \resumeSubHeadingListEnd
 

%----------- SKILLS -----------

\section{Skills}
  \resumeSubHeadingListStart
    \small{\item{
        \textbf{Programming:}{ Python, Java, R, \LaTeX, Octave, HTML, C/C++, Haskell, Unix scripting} \\ \vspace{2pt}
        
        \textbf{Frameworks:}{ TensorFlow, Keras, PyTorch, PaddlePaddle, NumPy, Jax, scikit-learn} \\ \vspace{2pt}
        
        \textbf{Cloud Computing:}{ AWS, Baidu AI Studio, Google Colab}
    }} \\ \vspace{-5pt}
  \resumeSubHeadingListEnd

  

%----------- PUBLICATIONS & MANUSCRIPTS -----------

\section{Publications \& Manuscripts}

  \resumeSubHeadingListStart
    \small{\item{
    \textcolor{gray}{(* denotes equal contributions)} \\ \vspace{2pt}
    	$\cdot$ \textbf{Wang, Z.} (2023). Probabilistic Linguistic Knowledge and Token-level Text Augmentation. \emph{Signals and Communication Technology (Springer)}. Book chapter. In Preparation. \\ \vspace{2pt}
	
    	$\cdot$ \textbf{Wang, Z.} (2023). \href{https://arxiv.org/abs/2303.06841}{Learning Transductions and Alignments with RNN Seq2seq Models.} Preprint. \emph{arXiv:2303.06841}. \\ \vspace{2pt}
    
        $\cdot$ Li, Z*., Makarova, V*. \& \textbf{Wang, Z*.} (2023). \href{https://www.frontiersin.org/articles/10.3389/fcomm.2023.1035394/full}{Developing Literature Review Writing Skills through an Online Writing Tutorial Series: Corpus-based Evidence.} \emph{Frontiers in Communication-Language Sciences}. \\ \vspace{2pt}

        $\cdot$ \textbf{Wang, Z.} (2022). \href{https://aclanthology.org/2022.icnlsp-1.5/}{Linguistic Knowledge in Data Augmentation for Natural Language Processing: An Example on Chinese Question Matching.} \emph{Proceedings of the 5th International Conference on Natural Language and Speech Processing}. \\ \vspace{2pt}
        
        $\cdot$ \textbf{Wang, Z.} (2022). \href{https://aclanthology.org/2022.eval4nlp-1.6/}{Random Text Perturbations Work, but not Always.}  \emph{Proceedings of the 3rd Workshop on Evaluation and Comparison of NLP Systems (co-located at AACL 2022)}. \\ \vspace{2pt}
        
        $\cdot$ Hao, H., Cui, Y., \textbf{Wang, Z.} \& Kim, Y. (2022). \href{https://ieeexplore.ieee.org/abstract/document/9903512}{Thirty-Two Years of IEEE VIS: Authors, Fields of Study and Citations.} \emph{IEEE Transactions on Visualization and Computer Graphics}.  \\ \vspace{2pt}
        
        $\cdot$ \textbf{Wang, Z.} (2021). \href{https://harvest.usask.ca/handle/10388/13387}{A Macroscopic Re-examination of Language and Gender: A Corpus-based Case Study in University Instructor Discourses.} University of Saskatchewan. \\ \vspace{2pt}
        
        $\cdot$ Peng, L*., \& \textbf{Wang, Z*.} (2017). \href{https://www.researchgate.net/publication/340730935_The_contemporary_reflection_on_the_Sentence-Based_theory_postulated_by_New_Grammar_of_Modern_Chinese}{The Contemporary Reflection on the Sentence-based Theory Postulated by New Grammar of Modern Chinese.} \emph{Journal of Hengyang Normal University}. 38(5), 95-101.
    }} 
  \resumeSubHeadingListEnd
  

%----------- PROJECTS -----------

\section{Projects}
    \resumeSubHeadingListStart
    
    \resumeProjectHeading
        {\textbf{RNN Seq2seq Models Learning Transductions and Alignments} $|$ \emph{\href{https://github.com/jaaack-wang/rnn-seq2seq-learning}{\color{blue}GitHub}} $|$ \emph{\href{https://arxiv.org/abs/2303.06841}{\color{blue}Manuscript}}}{2022 \textbf{--} 2023}
          \resumeItemListStart
            \resumeItem{Designed and conducted comprehensive experiments examining the capabilities of RNN seq2seq models in  learning four transduction tasks of varying complexity and that can be described as learning alignments}
          \resumeItemListEnd
    
    \resumeProjectHeading
        {\textbf{Thirty-two Years of IEEE VIS: Authors, Fields of Study and Citations} $|$ \emph{\href{https://github.com/hongtaoh/32vis}{\color{blue}GitHub}}
        $|$ \emph{\href{https://ieeexplore.ieee.org/abstract/document/9903512}{\color{blue}Paper}}
        }{2022}
          \resumeItemListStart
            \resumeItem{Helped build two text classifiers to predict missing affiliation and country data; analyzed VIS authors over the past 32 years and visualized the collaboration network; contributed to the core idea of visualizing temporal trends}
          \resumeItemListEnd
          
      \resumeProjectHeading
        {\textbf{Deep Learning Based Natural Language Processing using Paddlenlp} $|$ \emph{\href{https://github.com/jaaack-wang/dl-nlp-using-paddlenlp}{\color{blue}GitHub}}}{2021 \textbf{--} 2022}
          \resumeItemListStart
            \resumeItem{Finetuned pretrained models, such as BERT, RoBERTa, ERNIE, SKEP, on 11 NLP tasks (e.g., text similarity, sentiment analysis, named entity recognition, relation extraction, Q\&A system) to reproduce the SOTA results}
            % \resumeItem{Created tutorials on word embedding, and text (matching) classification tasks utilizing deep learning models}
          \resumeItemListEnd

    \newpage
      
            \resumeProjectHeading
        {\textbf{Linguistic Knowledge in Data Augmentation for Natural Language Processing} $|$ \emph{\href{https://github.com/jaaack-wang/linguistic-knowledge-in-DA-for-NLP}{\color{blue}GitHub}} 
        $|$ \emph{\href{https://aclanthology.org/2022.icnlsp-1.5/}{\color{blue}Paper}}}{2021}
          \resumeItemListStart
            \resumeItem{Conducted the first (cross-lingual) experiments that demonstrate the limitations of random text perturbations as text augmentation and the minimal role of probabilistic linguistic knowledge in the context of deep learning}
            \resumeItem{Designed two text augmentation programs, with or without a N-gram language model, that augment text with 5 token-level text editing operations and can be easily adapted for other languages}
          \resumeItemListEnd
      
      \resumeProjectHeading
        {\textbf{HELPtk: Historical English Language Processing Toolkit} $|$ \emph{\href{https://github.com/jaaack-wang/HELPtk}{\color{blue}GitHub}}}{2021}
          \resumeItemListStart
            \resumeItem{Created a general and open-ended framework that can tokenize, normalize \& annotate a normal XML corpus of a few million tokens in few minutes for historical English texts with improved accuracy applying Stanford CoreNLP}
            \resumeItem{Hand crafted a few hundred normalization and tokenization rules using Regular Expression, informed by the textual distribution of historical English texts discovered by naive Bayes method}
          \resumeItemListEnd
      
      \resumeProjectHeading
        {\textbf{A Macroscopic Re-examination of Language and Gender} $|$ \emph{\href{https://github.com/jaaack-wang/ling_feature_extractor}{\color{blue}GitHub}} 
        $|$ \emph{\href{https://harvest.usask.ca/handle/10388/13387}{\color{blue}Manuscript}}}{2020 \textbf{--} 2021}
          \resumeItemListStart
            \resumeItem{Provided the first comparative accounts of male and female university instructor discourses by macroscopically examining the use of 87 syntactic, lexical, and discoursal features in a large compiled corpus}
            \resumeItem{Developed a general-purpose corpus-linguistic tool to extract and search for linguistic features}
          \resumeItemListEnd
      
    \resumeSubHeadingListEnd
    
    
%----------- TUTORIALS -----------

\section{Tutorials}
    \resumeSubHeadingListStart
    
     \resumeProjectHeading
        {\textbf{rnn-transduction \& rnn-seq2seq-transduction} $|$ \emph{\href{https://github.com/jaaack-wang/rnn-transduction}{\color{blue}Tutorial1}}
        $|$ \emph{\href{https://github.com/jaaack-wang/rnn-seq2seq-transduction}{\color{blue}Tutorial2}}
        }{2023}
          \resumeItemListStart
            \resumeItem{Using RNN and RNN seq2seq models (in PyTorch) for modelling string transduction tasks.}
          \resumeItemListEnd
    
    \resumeProjectHeading
        {\textbf{Text classification \& text pair classification explained} $|$ \emph{\href{https://github.com/jaaack-wang/text-classification-explained}{\color{blue}Tutorial1}}
        $|$ \emph{\href{https://github.com/jaaack-wang/text-matching-explained}{\color{blue}Tutorial2}}
        }{2022}
          \resumeItemListStart
            \resumeItem{Building deep learning models for text (pair) classification from scratch using paddle, PyTorch, and TensorFlow}
          \resumeItemListEnd

    \resumeProjectHeading
        {\textbf{Notes for Stanford CS224N NLP with Deep Learning} $|$ \emph{\href{https://github.com/jaaack-wang/Notes-for-Stanford-CS224N-NLP-with-Deep-Learning}{\color{blue}GitHub}}
        }{2021}
          \resumeItemListStart
            \resumeItem{Notes cover the conceptual and mathematical basics of word embedding, neural networks, deep learning models}
            \resumeItem{Related tutorials of mine: \href{https://github.com/jaaack-wang/dl-nlp-using-paddlenlp/tree/main/paddlenlp_updated_notes_English/WordEmbedding}{\color{blue}Word Embedding}; \href{https://github.com/jaaack-wang/hands-on-gradients-derivation-for-ml-dl-loss-func}{\color{blue}Gradient Derivation for ML/DL Loss Functions}}
          \resumeItemListEnd

    \resumeSubHeadingListEnd
 
 \vspace{-10pt}

 %----------- PRESENTATION -----------

\section{Presentations}

  \resumeSubHeadingListStart
    \small{\item{
      
        $\cdot$ \textbf{Wang, Z.} (2023). Exploring RNN Seq2seq Models via Transduction Tasks that Require Alignments. The 10th Mid-Atlantic Student Colloquium on Speech, Language and Learning (MASC-SLL 2023), George Mason University. \\ \vspace{2pt}
        
        $\cdot$ \textbf{Wang, Z.} (2023). Probing the learning capabilities of RNN seq2seq models (Poster). IACS (Institute for Advanced Computational Science) Research Day, Stony Brook University. \\ \vspace{2pt}

        $\cdot$ \textbf{Wang, Z.} (2023). Probing the learning capabilities of RNN seq2seq models (Poster). The 47th Penn Linguistics Conference, University of Pennsylvania. \\ \vspace{2pt}

        $\cdot$ \textbf{Wang, Z.} (2023). Understanding how RNN seq2seq models learn alignment. SYNC (STONY BROOK, YALE, NYU, CUNY Linguistics Conference) 23, Yale University. \\ \vspace{2pt}
        
        $\cdot$ \textbf{Wang, Z.} (2022). Linguistic Knowledge in Data Augmentation for Natural Language Processing: An Example on Chinese Question Matching. The 5th International Conference on Natural Language and Speech Processing. \\ \vspace{2pt}
        
        $\cdot$ \textbf{Wang, Z.} (2022). Random Text Perturbations Work, but not Always. The 3rd Workshop on Evaluation and Comparison of NLP Systems (co-located at AACL 2022). \\ \vspace{2pt}


%        $\cdot$ \textbf{Wang, Z.}, Li, Z., \& Makarova, V. (2022) Don’ts of English Grammar in International Graduate Students’ ESL Academic Writing. Prairie Workshop on Languages and Linguistics VI, University of Saskatchewan. \\ \vspace{2pt}
%        
       $\cdot$ Makarova, V., Li, Z. \& \textbf{Wang, Z.} (2022) What international graduate students seek and find in an online Academic English Writing tutorial course. The Canadian Association of Applied Linguistics (ACLA/CAAL) 2022. \\ \vspace{2pt}
%  
        
        $\cdot$ \textbf{Wang, Z.}, Li, Z., \& Makarova, V. (2021) Developing an online academic writing tutorial for non-native English speaking international graduate students in diverse programs of studies. International Teaching Online Symposium, University of Windsor. \\ \vspace{2pt}
        
        $\cdot$ \textbf{Wang, Z.}, Tucker, B. V. \& Mukai, Y. (2018). How does Chinese learner of English perceive and comprehend phonetic reduction? Summer Poster Symposium at University of Alberta.
    }} 
  \resumeSubHeadingListEnd



% ----------- AWARDS & HONORS -----------

\section{Awards \& Honors}
  \resumeSubHeadingListStart
    \small{\item{
        \textbf{Graduate Thesis Award}{, University of Saskatchewan, 2021} \\ \vspace{3pt}
        
        \textbf{Mitacs Globalink Graduate Fellowship}{, Mitacs, 2019} \\ \vspace{3pt}
        
        \textbf{Excellent Graduate of the year}{, Hunan University, 2019} \\ \vspace{3pt}
        
        \textbf{Chinese Government Scholarship}{, China Scholarship Council, 2018} \\ \vspace{3pt}
        
        \textbf{The First-Class Scholarship}{, Hunan University, 2017, 2018} \\ \vspace{3pt}
        
        \textbf{National Student Innovation Training Program Grant}{, Ministry of Education of China, 2017} \\ \vspace{3pt}
        
        \textbf{Triple-A student}{, Hunan University, 2016} \\ \vspace{3pt}
        
        \textbf{National Scholarship}{, Ministry of Education of China, 2016}
    }}
  \resumeSubHeadingListEnd

%-------------------------------------------
\end{document}