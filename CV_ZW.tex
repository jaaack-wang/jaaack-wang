\documentclass[a4paper, 11pt]{article}  % there seems to be nuances between [] (attributes?) and {} (func names?)

\usepackage{hyperref} % for \url{_true_link_} and \href{_true_link_}{_display_name_}

\usepackage{geometry} % set margins of the document
 \geometry{
 a4paper,
 total={170mm, 257mm}, % left-right, top-down
 left=20mm,
 top=20mm,
 }

% ======================= Body Part Begins =======================

\begin{document} % whenever there is a begin, there is an end

\begin{center}

	\huge \textbf{Zhengxiang Wang} \vspace{11pt} \\  % \Huge is even huger than \huge; \\: breaks a new line
	\small \url{http://jaaack-wang.eu.org/} 

\end{center}

% ------------------------------------- Eduction -------------------------------------
\section*{EDUCATION} % * is for suppressing the indexed numbers for each section
\hrule % a horizontal line; default looks good
\vspace{11pt} % add an11 pt vertical space

\textbf{University of Saskatchewan}, Canada \hfill Sep 2019 – May 2021 % \hfill get texts in two ends of a line

\begin{itemize}
	\itemsep0em % this is to reduce the (vertical) spaces between listed items
	
	\item{Master of Arts in Applied Linguistics}
	\item{Thesis: A macroscopic re-examination of language and gender: A corpus-based case study in university instructor discourses [\href{https://harvest.usask.ca/bitstream/handle/10388/13387/WANG-THESIS-2021.pdf?sequence=1\&isAllowed=y}{PDF}]}
	
\end{itemize}

\vspace{11pt}

\noindent
\textbf{Hunan University}, China \hfill Sep 2015 – June 2019

\begin{itemize}
	\itemsep0em 
	
	\item{Bachelor of Arts in Chinese Language and Literature}
	
\end{itemize}


% ------------------------------------- Research -------------------------------------
\section*{RESEARCH EXPERIENCES}
\hrule 
\vspace{11pt}

\textbf{Graduate Research Assistant}, University of Saskatchewan \hfill May 2020 – Jun 2021 \vspace{3pt} \\ 
\emph{Supervised by Prof. Veronika Makarova, Prof. Zhi Li}

\begin{itemize}
	\itemsep0em 
	
	\item{Research topics: disfluency and unfilled pauses; academic English writing by L2 learners}
	\item{Roles: provided technical (e.g., programming, website development), recruitment, and project management supports; bibliographic research and manuscript writing; conference presentation}
	
\end{itemize}

\vspace{11pt}

\noindent
\textbf{Student Principal Investigator}, Hunan University \hfill Apr 2017 – May 2019 \vspace{3pt} \\ 
\emph{Supervised by Prof. Lanyu Peng}
\begin{itemize}
	\itemsep0em 
	
	\item{Project: Grammar as science: Rethinking the construction of Modern Chinese Grammar [\href{https://drive.google.com/file/d/1h_u2THzdFSRTRMPgM2UDXj7O_dGkJAN3/view}{PDF}] (in Chinese, 99 pages), funded by the National Student Innovation Training Program}
	
\end{itemize}

\vspace{11pt}

\noindent
\textbf{Research Intern}, University of Alberta \hfill Jun 2018 – Sep 2018 \vspace{3pt} \\ 
\emph{Supervised by Prof. Benjamin V. Tucker}
\begin{itemize}
	\itemsep0em 
	
	\item{Research topic: effects of English phonetic reduction on Chinese L2 learners [\href{https://drive.google.com/file/d/1Pq59XVx5zYgCazE93btu9G8RI6Mk_tVC/view}{Poster}]}
	
\end{itemize}

% ------------------------------------- PUBLICATIONS & MANUSCRIPTS -------------------------------------
\section*{PUBLICATION \& MANUSCRIPT} % & is a special token, use "\" to make it a normal one 
\hrule 
\vspace{11pt}

\begin{itemize}
	\itemsep0em 
	
	\item{\textbf{Wang, Z}. (2021). Linguistic Knowledge in Data Augmentation for Natural Language Processing: An Example on Chinese Question Matching. \emph{ArXiv}:2111.14709v2 [cs.CL]. [\href{https://arxiv.org/pdf/2111.14709.pdf}{PDF}]}
	\item{Peng, L, \& \textbf{Wang, Z}. (2017). The contemporary reflection on the Sentence-Based theory postulated by New Grammar of Modern Chinese. Journal of Hengyang Normal University, 38(5), 95-101. [\href{https://www.researchgate.net/publication/340730935_The_contemporary_reflection_on_the_Sentence-Based_theory_postulated_by_New_Grammar_of_Modern_Chinese}{PDF}]}
	
\end{itemize}


% ------------------------------------- SKILLS -------------------------------------
\section*{SKILLS}
\hrule 
\vspace{11pt}

\begin{tabular}{@{} l l } % {tabular} creates tabular; "@{}" ensures no indentation for the first column; "l l": two columns, left-aligned

 	\textbf{Programming}  & Python (Advanced), R (Fluent), Java (Fluent), MATLAB/Octave (Fluent), \\
	 	& JavaScript (Elementary), C (Elementary) \vspace{11pt} \\ 

 	\textbf{Markup Languages} & HTML, Markdown, XML, \LaTeX \vspace{11pt}  \\  

\end{tabular}

% ======================= Body Part Ends =======================

\end{document}