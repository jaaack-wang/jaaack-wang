%-------------------------------------------------------------------------------------------------
% Jack's CV in Latex. 
% I am new to Latex, so I am trying to learn it by simply
% creating a Latex's CV that resembles one I made in Word. 
% I had background in Markdown, HTML, and XML, so that helped me a lot. 
% Besides, my knowledge of Latex for this CV mainly comes from:
% - https://github.com/luong-komorebi/Begin-Latex-in-minutes
% - Google search (mostly from: https://www.overleaf.com/learn). 
%-------------------------------------------------------------------------------------------------

\documentclass[a4paper, 11pt]{article}  % there seems to be nuances between [] (attributes?) and {} (func names?)

\usepackage{hyperref} % for \url{_true_link_} and \href{_true_link_}{_display_name_}

\usepackage{geometry} % set margins of the document
 \geometry{
 a4paper,
 total={170mm, 257mm}, % left-right, top-down
 left=20mm,
 top=20mm,
 }

% ======================= Body Part Begins =======================

\begin{document} % whenever there is a begin, there is an end

\begin{center}

	\huge \textbf{Zhengxiang Wang} \vspace{11pt} \\  % \Huge is even huger than \huge; \\: breaks a new line
	\small \url{https://jaaack-wang.github.io/jaaack-wang/} 

\end{center}

% ------------------------------------- Eduction -------------------------------------
\section*{EDUCATION} % * is for suppressing the indexed numbers for each section
\hrule % a horizontal line; default looks good
\vspace{11pt} % add an11 pt vertical space

\textbf{University of Saskatchewan}, Canada \hfill Sep 2019 – May 2021 % \hfill get texts in two ends of a line

\begin{itemize}
	\itemsep0em % this is to reduce the (vertical) spaces between listed items
	
	\item{Master of Arts in Applied Linguistics; average score: 88/100}
	\item{Thesis: A macroscopic re-examination of language and gender: A corpus-based case study in university instructor discourses [\href{https://harvest.usask.ca/bitstream/handle/10388/13387/WANG-THESIS-2021.pdf?sequence=1\&isAllowed=y}{PDF}] (Nominated for Thesis Award, in process)}
	
\end{itemize}

\vspace{11pt}

\noindent
\textbf{Hunan University}, China \hfill Sep 2015 – June 2019

\begin{itemize}
	\itemsep0em 
	
	\item{Bachelor of Arts in Chinese Language and Literature; GPA: 3.66/4; \emph{summa cum laude}}
	\item{Thesis: Analysis on the quantitativeness of reduplication in Modern Mandarin (score: 92/100)}
	
\end{itemize}


% ------------------------------------- Research -------------------------------------
\section*{FORMAL RESEARCH TRAINING}
\hrule 
\vspace{11pt}

\textbf{Graduate Research Assistant}, University of Saskatchewan \hfill May 2020 – Jun 2021 \vspace{3pt} \\ 
\emph{Supervised by Prof. Veronika Makarova, Prof. Zhi Li}

\begin{itemize}
	\itemsep0em 
	
	\item{Research topics: disfluency and unfilled pauses; academic English writing by L2 learners}
	\item{Roles: provided technical (e.g., programming, website development), recruitment, and project management supports; bibliographic research and manuscript writing; conference presentation}
	
\end{itemize}

\vspace{11pt}

\noindent
\textbf{Student Principal Investigator}, Hunan University \hfill Apr 2017 – May 2019 \vspace{3pt} \\ 
\emph{Supervised by Prof. Lanyu Peng}
\begin{itemize}
	\itemsep0em 
	
	\item{Project: Grammar as science: Rethinking the construction of Modern Chinese Grammar [\href{https://drive.google.com/file/d/1h_u2THzdFSRTRMPgM2UDXj7O_dGkJAN3/view}{PDF}] (in Chinese, 99 pages), funded by the National Student Innovation Training Program}
	\item{Outcomes: 11 essays/papers with one published; first-prize in the final project presentation}
	
\end{itemize}

\vspace{11pt}

\noindent
\textbf{Research Intern}, University of Alberta \hfill Jun 2018 – Sep 2018 \vspace{3pt} \\ 
\emph{Supervised by Prof. Benjamin V. Tucker}
\begin{itemize}
	\itemsep0em 
	
	\item{Research topic: effects of English phonetic reduction on Chinese L2 learners [\href{https://drive.google.com/file/d/1Pq59XVx5zYgCazE93btu9G8RI6Mk_tVC/view}{Poster}]}
	\item{Roles: used Praat to annotate and segment phonetically reduced words in a corpus of spontaneous English as stimuli; designed and run a psycholinguistic experiment with E-prime 2}
	
\end{itemize}

% ------------------------------------- PUBLICATIONS & MANUSCRIPTS -------------------------------------
\section*{PUBLICATIONS \& MANUSCRIPTS} % & is a special token, use "\" to make it a normal one 
\hrule 
\vspace{11pt}

\begin{itemize}
	\itemsep0em 
	
	\item{Li, Z, Makarova, V \& \textbf{Wang, Z}. (2022). Developing literature review writing skills through an online writing tutorial series: Corpus-based evidence. Frontiers. Abstract accepted.}
	\item{\textbf{Wang, Z}. (2021). Linguistic Knowledge in Data Augmentation for Natural Language Processing: An Example on Chinese Question Matching. \emph{ArXiv}:2111.14709v2 [cs.CL]. [\href{https://arxiv.org/pdf/2111.14709.pdf}{PDF}]}
	\item{\textbf{Wang, Z}, Li, Z \& Zhou, Z. A proposal for unfilled pause reclassification in spontaneous English speech. Manuscript in preparation.}
	\item{Peng, L, \& \textbf{Wang, Z}. (2017). The contemporary reflection on the Sentence-Based theory postulated by New Grammar of Modern Chinese. Journal of Hengyang Normal University, 38(5), 95-101. [\href{https://www.researchgate.net/publication/340730935_The_contemporary_reflection_on_the_Sentence-Based_theory_postulated_by_New_Grammar_of_Modern_Chinese}{PDF}]}
	
\end{itemize}

% -------------------------------------  PRESENTATIONS -------------------------------------
\section*{PRESENTATIONS}
\hrule 
\vspace{11pt}

\begin{itemize}
	\itemsep0em 
	
	\item{\textbf{Wang, Z}, Li, Z \& Makarova, V. (2021). Developing an online academic writing tutorial for non-native English speaking international graduate students in diverse programs of studies. International Teaching Online Symposium, University of Windsor.}
		
	\item{\textbf{Wang, Z}, Tucker, B. V. \& Mukai, Y. (2018). How does Chinese learner of English perceive and comprehend phonetic reduction? Summer Poster Symposium at University of Alberta.}
	
\end{itemize}


% ------------------------------------- TEACHING & MENTORING -------------------------------------
\section*{TEACHING \& MENTORING }
\hrule 
\vspace{11pt}

\textbf{Research Mentor}, Mitacs \hfill Apr 2021 – Sep 2021
\begin{itemize}
	\itemsep0em 
	
	\item{Mentored 9 visiting undergrad researchers to University of Saskatchewan}
	\item{Answered questions, explained polices, and organized virtual social events}
	
\end{itemize}

\vspace{11pt}

\noindent
\textbf{Online English Tutor}, Changjun Long-distance Education \hfill Mar 2017 – May 2019 
\begin{itemize}
	\itemsep0em 
	
	\item{Taught English to high-school students in the form of pre-recorded videos}
	\item{Created assignments; responded to students’ questions; interacted with parents}
	
\end{itemize}

% ------------------------------------- SKILLS -------------------------------------
\section*{SKILLS}
\hrule 
\vspace{11pt}

\begin{tabular}{@{} l l } % {tabular} creates tabular; "@{}" ensures no indentation for the first column; "l l": two columns, left-aligned

 	\textbf{Programming}  & Python (Advanced), R (Fluent), Java (Fluent), MATLAB/Octave (Fluent), \\
	 	& JavaScript (Elementary), C (Elementary) \vspace{11pt} \\ 

 	\textbf{Markup Languages} & HTML, Markdown, XML, \LaTeX \vspace{11pt}  \\  
	
 	\textbf{Linguistic Tools} & Praat, E-Prime, autosub (auto speech-to-text transcription), \\
	 	& cloud translation (Google, Baidu, Papago etc.), LanguageTool (writing checker) \vspace{11pt} \\ 
	
	\textbf{Languages} & Mandarin (Native), Fuqing dialect (Native), English (Advanced), \\ 
	 	&  Korean (Elementary), classic Chinese reading \& writing (Advanced), \\
		& Shandong dialects (fieldwork), Media Lengua and Quichua (coursework)

\end{tabular}

% ------------------------------------- NLP RESOURCES -------------------------------------
\section*{NLP RESOURCES}
\hrule 
\vspace{11pt}

\begin{tabular}{@{} l l }
	
	\textbf{Deep Learning}  &  Text matching (\href{https://github.com/jaaack-wang/text-matching-explained}{Tutorials}); Text classification (\href{https://github.com/jaaack-wang/text-classification-explained}{Tutorials}) \\
		& Notes for Stanford CS224N NLP with Deep Learning (\href{https://github.com/jaaack-wang/Notes-for-Stanford-CS224N-NLP-with-Deep-Learning}{Notes}); \\ 
		& Deep Learning Based NLP using paddlenlp (\href{https://github.com/jaaack-wang/dl-nlp-using-paddlenlp}{Notes});  \\ 
		& Word Embedding (\href{https://github.com/jaaack-wang/dl-nlp-using-paddlenlp/tree/main/paddlenlp_updated_notes_English/WordEmbedding}{Tutorials}); Gradient Derivation (\href{https://github.com/jaaack-wang/hands-on-gradients-derivation-for-ml-dl-loss-func}{Tutorials}) \vspace{11pt} \\
		
	\textbf{Text processing} & Text Augmentation techniques (\href{https://github.com/jaaack-wang/text-augmentation-techniques}{Source Code}); \\ 
		& Historical English Language Processing Toolkit (\href{https://github.com/jaaack-wang/HELPtk}{Source Code}); \\ 
		& Linguistic Feature Extractor (\href{https://github.com/jaaack-wang/ling_feature_extractor}{Source Code}); \\ 
		& Unfilled Pause Classifier (\href{https://github.com/jaaack-wang/BASE_SLN_Pause_Project/blob/main/Scripts/pause_type_automation.py}{Source Code}) \vspace{11pt} \\

	\textbf{Web Scraping} & Google Scholar Analyzer (\href{https://github.com/jaaack-wang/GSchoolarAnalyzer}{Source Code}); \\ 
		& YouTube Info Collector (\href{https://github.com/jaaack-wang/YouTubeInfoCollector}{Source Code})
	
\end{tabular}

% ------------------------------------- CERTIFIED COURSES & TRAINING -------------------------------------
\section*{CERTIFIED COURSES \& TRAINING}
\hrule 
\vspace{11pt}

Deep Learning Specialization (Deeplearning.AI); Machine Learning (Stanford Online); Deep Learning Based Natural Language Processing (Baidu PaddlePaddle); Linear Algebra (Imperial College London); Multivariate Calculus (Imperial College London); Introduction to Calculus (University of Sydney).

% ------------------------------------- COMMUNITY SERVICES -------------------------------------
\section*{COMMUNITY SERVICES}
\hrule 
\vspace{11pt}

\textbf{ESL classroom volunteer},  The Global Gathering Place, Saskatoon \hfill Dec 2019 – Mar 2020
\begin{itemize}
	\itemsep0em 
	
	\item{Provided in-classroom supports for adult immigrants to learn English}
	
\end{itemize}

\vspace{11pt}

\noindent
\textbf{Organizer and Volunteer}, Univ. of Regina Study-in-China program, HNU \hfill May 2019 – Jun 2019
\begin{itemize}
	\itemsep0em 
	
	\item{Arranged student volunteers and organized reception events}
	\item{Helped one visually impaired Canadian student with life and Mandarin}
	
\end{itemize}

\vspace{11pt}

\noindent
\textbf{Interpreter}, UNESCO conference “Role of linguistic diversity”, Changsha \hfill Sep 2018
\begin{itemize}
	\itemsep0em 
	
	\item{Served as an interpreter and local guide for a Japanese professor}
	
\end{itemize}

% ------------------------------------- AWARDS & HONORS -------------------------------------
\section*{AWARDS \& HONORS}
\hrule 
\vspace{11pt}

\begin{itemize}
	\itemsep0em 
	
	\item{John Spencer Middleton \& Jack Spencer Gordon Middleton Graduate Award, 2020}
	\item{Mitacs Globalink Graduate Fellowship, Mitacs, 2019.}
	\item{Excellent Graduate of the year, Hunan University, 2019}
	\item{Chinese Government Scholarship, China Scholarship Council, 2018}
	\item{The First-Class Scholarship, Hunan University, 2017, 2018}
	\item{National Student Innovation Training Program Grant, Ministry of Education of China, 2017}
	\item{Triple-A student, Hunan University, 2016}
	\item{National Scholarship, Ministry of Education of China, 2016}
	
\end{itemize}

% ======================= Body Part Ends =======================

\end{document}